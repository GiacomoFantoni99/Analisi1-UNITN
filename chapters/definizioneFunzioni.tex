\chapter{Funzioni}
\section{Definizione}
Dati due insiemi $\mathbb{X},\mathbb{Y}$ qualsiasi, una funzione di dominio $\mathbb{X}$ e valori in $\mathbb{Y}$ (=codominio) è una qualsiasi legge che ad ogni elemento di $\mathbb{X}$ associa uno (e \textbf{uno solo}) elemento di $\mathbb{Y}$.
\begin{center}
	\begin{LARGE}
		$f: \mathbb{X} \rightarrow \mathbb{Y}$\\
	\end{LARGE}
	$x \rightarrow y=f(x)$
\end{center}

\section{Funzioni particolari}
\subsection{Funzione costante}
\begin{Large}
$\bar{y} \in \mathbb{Y}, f: \mathbb{X} \rightarrow \mathbb{Y}\\
f(x)=\bar{y}\ \forall\ x \in \mathbb{X}$
\end{Large}
\subsection{Funzione identità}
\begin{Large}
$I_x: \mathbb{X} \rightarrow \mathbb{X}\\
I_x(x)=x\ \forall\ x \in \mathbb{X}$
\end{Large}
\subsection{Funzione restrizione}
\begin{Large}
$f: \mathbb{X} \rightarrow \mathbb{Y}; \mathbb{A} \subseteq \mathbb{Y}\\
f\restriction_\mathbb{A} =f(x)\ \forall\ x \in \mathbb{A}\\
f\restriction_\mathbb{A}: \mathbb{A} \rightarrow \mathbb{Y};\ x \rightarrow f(x)$
\end{Large}

\section{Insieme Immagine}
Sia $f: \mathbb{X} \rightarrow \mathbb{Y}$ e $\mathbb{A} \subseteq \mathbb{Y}$, diremo $\mathbb{A}$ tramite $f$ l'insieme $f(\mathbb{A}) = \{f(x) | x \in \mathbb{A}\}$
Se $\mathbb{A} = \mathbb{X}$, $f(\mathbb{X})$ è semplicemente detta Immagine di $f$; $f(\mathbb{X}) = Imf$

\section{Iniettività, Suriettività e Biettività}
Sia $f: \mathbb{X} \rightarrow \mathbb{Y}$
\subsection{Iniettività}
Se: \begin{Large}
$\forall x_1,x_2 \in \mathbb{X}, x_1 \neq x_2, f(x_1) \neq f(x_2)$ 
\end{Large} 
allora la funzione $f$ è una funzione \textbf{Iniettiva}.\\
\\
Def 2:\\
\begin{Large}
$f(x_1) = f(x_2) \iff x_1=x_2$ 
\end{Large}

\subsection{Suriettività}
Se: \begin{Large}
$\forall y \in \mathbb{Y}\ \exists\ x \in \mathbb{X}\ |\ f(x) = y$ 
\end{Large} 
allora la funzione $f$ è una funzione \textbf{Suriettiva}.\\
\\
Def 2:\\
\begin{Large}
$Imf = \mathbb{Y}$ 
\end{Large}

\subsection{Biettività}
La funzione $f$ è \textbf{Biettiva} $\iff$ $f$ è sia iniettiva che suriettiva.\\
\begin{Large}
$f: \mathbb{X} \rightarrow \mathbb{Y}$ biettiva $\iff \forall\ y \in \mathbb{Y}\ \exists !\ x \in \mathbb{X} | f(x)=y$
\end{Large}

\section{Grafico}
Sia $f: \mathbb{X} \rightarrow \mathbb{Y}$; si dice grafico di $f$ (indicato con $G(f)$ o $graph(f)$) il sottoinsieme di $\mathbb{X}\times\mathbb{Y}$ definito come:\\
\begin{Large}
$graph(f)=\{(x,f(x)), x \in \mathbb{X}\} \subseteq \mathbb{X}\times\mathbb{Y}$
\end{Large}\\
\\
Spesso si usa improriamente la parola \textit{funzione} per indicare il \textit{grafico di funzione}, ma è generalmente accettato a scopo semplificazione del discorso

\section{Altre funzioni particolari}
\subsection{Funzione parte intera}
$[\cdot]: \mathbb{R} \rightarrow \mathbb{R}\\
x \rightarrow [x] = max\{n \in \mathbb{Z} | n\leq x\}$.
\subsection{Funzione di Heaviside}
$H: \mathbb{R} \rightarrow \mathbb{R}\\
x \rightarrow H(x) =
\begin{cases}
1 \text{  se } x \geq 0\\
0 \text{  se } x < 0
\end{cases}$
\subsection{Funzione segno}
$sgn: \mathbb{R} \rightarrow \mathbb{R}\\
x \rightarrow sgn(x) =
\begin{cases}
1 \text{  se } x > 0\\
0 \text{  se } x = 0\\
-1 \text{  se } x < 0
\end{cases}$
\subsection{Funzione mantissa}
$f: \mathbb{R} \rightarrow \mathbb{R}\\
x \rightarrow f(x) = x - [x]$

\section{Proprietà}
\subsection{Monotonia}
Sia $\mathbb{A} \subseteq \mathbb{X} \subseteq \mathbb{R}$ e $f: \mathbb{X} \rightarrow \mathbb{R}$ funzione detta (nell'insieme $\mathbb{A}$):
\begin{enumerate}
\item \textbf{crescente} se $\forall\ x_1,x_2 \in \mathbb{A}$, $x_1<x_2$; $f(x_1) \leq f(x_2)$
\item \textbf{strettamente crescente} se $\forall\ x_1,x_2 \in \mathbb{A}$, $x_1<x_2$; $f(x_1) < f(x_2)$
\item \textbf{decrescente} se $\forall\ x_1,x_2 \in \mathbb{A}$, $x_1<x_2$; $f(x_1) \geq f(x_2)$
\item \textbf{strettamente decrescente} se $\forall\ x_1,x_2 \in \mathbb{A}$, $x_1<x_2$; $f(x_1) > f(x_2)$
\item \textbf{monotona} Se $f$ crescente oppure decrescente
\item \textbf{strettamente monotona} Se $f$ strettamente crescente oppure strettamente decrescente
\end{enumerate}
\subsubsection{Osservazioni}
\begin{enumerate}
\item La somma di due funzioni crescenti è crescente;\\
La somma di due funzioni decrescenti è decrescente
\item Il prodotto di due funzioni crescenti non negative è crescente
\end{enumerate}

\subsection{Parità}
Sia $\mathbb{X}$ \textbf{simmetrico} rispetto l'origine (cioè: $\forall\ x \in \mathbb{X}$; $-x \in \mathbb{X}$); se $f: \mathbb{X} \rightarrow \mathbb{R}$ allora:
\begin{enumerate}
\item[i.] $f$ si dice \textbf{pari} in $\mathbb{X}$ se $f(x) = f(-x)\ \forall\ x \in \mathbb{X}$
\item[ii.] $f$ si dice \textbf{dispari} in $\mathbb{X}$ se $f(x) = -f(-x)\ \forall\ x \in \mathbb{X}$
\end{enumerate}

\subsection{Periodicità}
Sia $\mathbb{X} \subseteq \mathbb{R}$;   $f: \mathbb{X} \rightarrow \mathbb{R}$ si dice \textbf{periodica} di periodo $T>0$ se $T$ è il più piccolo numero reale tale che $x + T \in \mathbb{X}$, $f(x) = f(x + T)$\\
Ogni intervallo di lunghezza $T$ è detto intervallo di periodicità

\section{Estremi di funzione}
Sia $f: \mathbb{X} \rightarrow \mathbb{R}$
\subsection{Positività}
\begin{enumerate}
\item $f$ si dice \textbf{positiva} in $\mathbb{A} \subseteq \mathbb{R} \iff f(x)>0\ \forall\ x \in \mathbb{A}$
\item $f$ si dice \textbf{non negativa} in $\mathbb{A} \subseteq \mathbb{R} \iff f(x) \geq 0\ \forall\ x \in \mathbb{A}$
\item $f$ si dice \textbf{negativa} in $\mathbb{A} \subseteq \mathbb{R} \iff f(x)<0\ \forall\ x \in \mathbb{A}$
\item $f$ si dice \textbf{non positiva} in $\mathbb{A} \subseteq \mathbb{R} \iff f(x) \leq 0\ \forall\ x \in \mathbb{A}$
\end{enumerate}
\subsection{Limitazione, Massimo assoluto, Minimo assoluto, Estremo superiore e inferiore e caratterizzazione}
Gli estremi assoluti di $f$ sono gli stessi dell'insieme $f(\mathbb{X}) = Immf$; per le definizioni fare quindi riferimento agli \hyperref[sec: estremiInsiemi]{estremi di insiemi}
\subsection{Massimo relativo e minimo relativo}
TBD
\section{Successioni}
La funzione $f: \mathbb{N} \rightarrow \mathbb{R}$;  $n \rightarrow f(n) = a_n$ è chiamata successione e scritta anche come: $\{a_n\}_{n \in \mathbb{N}}$

\section{Composizione di funzioni}
Siano: $f: domf \rightarrow \mathbb{Y}$; $g: domg \rightarrow \mathbb{W}$ due funzioni e $\mathbb{A}=\{x \in domf | f(x) \in domg\}$. Si può allora definire:\\
\begin{Large}
$(g \circ f)(x) = g(f(x))$
\end{Large}\\
$g \circ f: \mathbb{A} \in \mathbb{W}$\\
In particolare $\mathbb{A} \subseteq \mathbb{Y}$ ed è sempre definita $g \circ f \restriction _\mathbb{A}$
\subsubsection{Osservazione}
$f \circ g \neq g \circ f$


\subsection{Proprietà}
Sia $\mathbb{A} \subseteq dom(g \circ f)$.
\begin{enumerate}
\item[i.] $\begin{cases}
f \text{ crescente in } \mathbb{A}\\
g \text{ crescente in } f(\mathbb{A})
\end{cases} \implies g \circ f \text{ crescente}$
\item[ii.] $\begin{cases}
f \text{ crescente in } \mathbb{A}\\
g \text{ decrescente in } f(\mathbb{A})
\end{cases} \implies g \circ f \text{ decrescente}$
\item[iii.] $\begin{cases}
f \text{ decrescente in } \mathbb{A}\\
g \text{ crescente in } f(\mathbb{A})
\end{cases} \implies g \circ f \text{ decrescente}$
\item[iv.] $\begin{cases}
f \text{ decrescente in } \mathbb{A}\\
g \text{ decrescente in } f(\mathbb{A})
\end{cases} \implies g \circ f \text{ crescente}$
\end{enumerate} 
