\chapter{Estremi di insiemi}
\label{sec: estremiInsiemi}
\section{Definizioni}
Sia $\mathbb{A} \subseteq \mathbb{R}$.
\subsection{Maggiorante}
Si dice che $\mathbb{A}$ è \textbf{superiormente limitato} se:\\
\begin{Large}
$\exists\ M \in \mathbb{R}\ |\ x \leq M\ \forall\ x \in \mathbb{A}$
\end{Large}\\
$M$ viene chiamato \textbf{maggiorante}
\subsection{Minorante}
Si dice che $\mathbb{A}$ è \textbf{inferiormente limitato} se:\\
\begin{Large}
$\exists\ m \in \mathbb{R}\ |\ x \geq m\ \forall\ x \in \mathbb{A}$
\end{Large}\\
$m$ viene chiamato \textbf{minorante}
\subsection{Massimo}
$max\mathbb{A}=\bar{x} \in \mathbb{R}$ è \textbf{massimo} di $\mathbb{A}$ se:\\
\begin{Large}
$\begin{cases}
x \leq \bar{x}\ \forall\ x \in \mathbb{A}\\
\bar{x} \in \mathbb{A}
\end{cases}$
\end{Large}
\subsection{Minimo}
$min\mathbb{A}=\hat{x} \in \mathbb{R}$ è \textbf{minimo} di $\mathbb{A}$ se:\\
\begin{Large}
$\begin{cases}
x \geq \hat{x}\ \forall\ x \in \mathbb{A}\\
\hat{x} \in \mathbb{A}
\end{cases}$
\end{Large}
\subsection{Osservazioni}
\begin{enumerate}
\item Se $\exists\ max\mathbb{A}$ o $\exists\ min\mathbb{A}$, essi sono \textbf{unici}
\item $\exists\ max\mathbb{A} \iff \exists\ M \in \mathbb{R}\ |\ x \leq M\ \forall\ x \in \mathbb{A}\\
\exists\ min\mathbb{A} \iff \exists\ m \in \mathbb{R}\ |\ x \geq m\ \forall\ x \in \mathbb{A}$
\end{enumerate}
\subsection{Estremo superiore}
Viene definito \textbf{estremo superiore} di A:\\
\begin{Large}
$sup\mathbb{A} = min\{M \in \mathbb{R} | x \leq M\ \forall\ x \in \mathbb{A}\}$
\end{Large}
\subsection{Estremo inferiore}
Viene definito \textbf{estremo inferiore} di A:\\
\begin{Large}
$inf\mathbb{A} = max\{m \in \mathbb{R} | x \geq m\ \forall\ x \in \mathbb{A}\}$
\end{Large}
\subsection{Osservazioni}
\begin{enumerate}
\item Se $\exists\ max\mathbb{A} \implies \exists\ supA=max\mathbb{A}\\
$Se $\exists\ minA \implies \exists\ infA=min\mathbb{A}$
\item Se $\exists\ sup\mathbb{A}, sup\mathbb{A} \in \mathbb{A} \implies \exists max\mathbb{A} = sup\mathbb{A}$\\
Se $\exists\ inf\mathbb{A}, inf\mathbb{A} \in \mathbb{A} \implies \exists min\mathbb{A} = inf\mathbb{A}$
\end{enumerate}

\section{Insiemi finiti}
$\mathbb{A}$ si dice \textbf{finito} se ha un numero finito di elementi.\\
Se $\mathbb{A} \subset \mathbb{R}$, $\mathbb{A}$ finito, $\mathbb{A} \neq \emptyset \implies \exists\ max\mathbb{A}, min\mathbb{A}$

\section{Caratterizzazione sup e inf}
\subsubsection{Caratterizzazione sup}
Sia $\mathbb{A} \subset \mathbb{R}$ superiormente limitato, $s \in \mathbb{R}$:\\
\begin{Large}
$s=sup\mathbb{A} \iff
\begin{cases}
s \geq x\ \forall\ x \in \mathbb{A}\\
\forall\ \epsilon > 0\ \exists\ x \in \mathbb{A}\ |\ x > s-\epsilon
\end{cases}$
\end{Large}
\subsubsection{Caratterizzazione inf}
Sia $\mathbb{A} \subset \mathbb{R}$ inferiormente limitato, $t \in \mathbb{R}$:\\
\begin{Large}
$t=inf\mathbb{A} \iff
\begin{cases}
t \leq x\ \forall\ x \in \mathbb{A}\\
\forall\ \epsilon > 0\ \exists\ x \in \mathbb{A}\ |\ x < t+\epsilon
\end{cases}$
\end{Large}

\label{sec: CompletezzaReali}
\section{Completezza dell'insieme dei Reali}
Sia $\mathbb{A} \subset \mathbb{R}$, $\mathbb{A} \neq \emptyset$:\\
\begin{Large}
Se $\mathbb{A}$ è limitato superiormente (o inferiormente) $\implies \exists\ sup\mathbb{A} \in \mathbb{R}$ (o $\exists\ inf\mathbb{A} \in \mathbb{R}$)
\end{Large}