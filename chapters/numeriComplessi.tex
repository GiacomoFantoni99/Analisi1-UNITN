\chapter{Numeri Complessi}
\section{Definizione}
Viene definito unità immaginaria $i$ quel numero per cui $i^2 = -1$.
L'insieme dei numeri complessi è quindi:\\
\begin{Large}
$\mathbb{C} = \{z = x + iy;\ x,y \in \mathbb{R}\}$
\end{Large}\\
ove $x + iy$ è la scrittura algebrica di un numero complesso.\\
In particolare $x$ è chiamata parte reale del numero: $x = Rez$; mentre $y$ parte immaginaria $y = Imz$.\\
\subsubsection{Osservazione}
$z \in \mathbb{C}$ è reale ($z \in \mathbb{R}$) $\iff Imz = 0$

\section{Piano di Gauss}
Il piano di Gauss è un piano $\mathbb{R}^2$ ove l'asse delle ascisse è $Rez$ e l'asse delle ordinate $Imz$\\
Ogni numero complesso $z$ è quindi rappresentabile come un punto sul piano di Gauss\\
---------------TBD: INSERIRE IMMAGINE PIANO DI GAUSS--------------------

\section{Alcune operazioni}
$z_1,z_2 \in \mathbb{C}$
\subsection{Somma}
\begin{Large}
$z_1+z_2 = x_1+x_2+i(y_1+y_2)$ 
\end{Large}\\
$Re(z_1+z_2) = x_1+x_2$;   $Im(z_1+z_2) = y_1+y_2$
\subsection{Prodotto}
\begin{Large}
$z_1z_2 = x_1x_2-y_1y_2+i(x_1y_2+ x_2y_1)$ 
\end{Large}\\
$Re(z_1z_2) = x_1x_2-y_1y_2$;   $Im(z_1z_2) = x_1y_2 + x_2y_1$
\subsection{Modulo}
$z = x+iy \in \mathbb{C}$\\
Modulo di $z$:\\
\begin{Large}
$|z| = \sqrt{x^2+y^2} \in \mathbb{R}$
\end{Large}\\
Nel piano di Gauss rappresenta la distanza dall'origine

\section{Numeri complessi notevoli}
$z \in \mathbb{C}$
\subsubsection{Elemento neutro somma}
$0\ +\ i0$
\subsubsection{Elemento opposto somma}
$-x\ -\ iy$
\subsubsection{Elemento neutro prodotto}
$1\ +\ i0$
\subsubsection{Elemento reciproco prodotto}
\begin{Large}
$\frac{1}{z}=\frac{1}{x+iy}=\frac{x-iy}{x^2+y^2}$
\end{Large}

\subsection{Coniugato di z}
$\bar{z} = x-iy$
\subsubsection{Osservazioni}
$z_1,z_2 \in \mathbb{C}$\\
$\overline{1+z_2} = \overline{z_1}+\overline{z_2}\\
\overline{z_1z_2} = \overline{z_1}\overline{z_2}$\\
\\
\begin{Large}
$|\bar{z}| = |z|\\
z\bar{z}=|z|^2\\
\frac{1}{z} = \frac{\bar{z}}{|z|^2}$
\end{Large}
