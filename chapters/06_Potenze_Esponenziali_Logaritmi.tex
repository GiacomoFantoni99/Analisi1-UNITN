\chapter{Potenze, esponenziali, logaritmi}
\section{Potenze}
\subsection{Propriet\`a}
\begin{itemize}
\item $x^n\cdot x^m =x^{m+n}$
\item $(x^n)^m=x^{m\cdot n}$
\item $(xy)^n=x^n\cdot y^n$
\end{itemize}
\subsection{Potenze ad esponente intero positivo $x^n$}
$\forall x \in \mathbf{R}$
\begin{itemize}
\item$x^2=x\cdot x$
\item$x^3=x\cdot x\cdot x$
\item$x^n=x^{n-1}\cdot x$
\item$x^n \ge 0 \forall x \in \mathbf{R}$, n pari.
\item$x^n \le 0 \forall x \in \mathbf{R^-}$, n dispari.
\end{itemize}
Se n \`e 2, $\forall x_1, x_2, \in \mathbf{R}, 0<x_1<x_2 \Rightarrow {x_1}^2 < {x_2}^2$. Si dimostra moltiplicando prima per $x_1$ e poi per $x_2$ e confrontando le due 
equazioni: $\cdot x_1= x_1\cdot x_1 < x_2\cdot x_1$,\\ $\cdot x_2= x_1\cdot x_2 < x_2\cdot x_2 $,\\ essendo $x_1\cdot x_2 < x_2\cdot x_1$,\\ ottengo che $x_1\cdot x_1 < x_2\cdot x_2 \Rightarrow {x_1}^2 < {x_2}^2$.\\ Se $x_1<x_2<0 \Rightarrow {x_1}^2 > {x_2}^2 $.\\ Se n\`e 3 $x_1<x_2 \Rightarrow  {x_1}^3 < {x_2}^3$
\subsection{Potenze ad esponente intero negativo $x^-n$}
$\forall x \in \mathbf{R}\backslash \{0\}$\\
$x^{-n}=\frac{1}{x^n}$
\subsection{Potenze ad esponente frazionario $x^\frac{1}{n}$}
\subsubsection{Esistenza della radice n-esima}
Sia $y \in \mathbf{R}, y\ge 0, n \in \mathbf{N}n n\ge 1 \Rightarrow \exists ! x \in \mathbf{R}, x \ge 0: x^n=y$.\\
x \`e indicato come $y^\frac{1}{n}$ no $\sqrt[n]{y}$\\
$\sqrt[n]{y}=sup\{a \in [0, +\infty[:a^n\le y\}$
\subsection{Potenze ad esponente razionale $y^q$}
$\forall x \in \mathbf{R}, x\ge 0, x^q\doteq (x^\frac{1}{n})^m=(x^m)^\frac{1}{n}, \forall q=\frac{m}{n}, m,n \in \mathbf{N}, n\neq 0$
\\
Si consideri $x \in \mathbf{R^+}, x=a$, mentre gli esponenti $q, r, s \in \mathbf{Q}$. \`E ben definito $a^r, r\in \mathbf{Q}$. Dalle propriet\`a delle potenze segue che
$\forall a, b \in \mathbf{R^+}, \forall r,s \in \mathbf{Q}$ esse si conservano, inoltre $a^r>0$,$a^0=1$\\
Se $(a>1 \wedge r>0)\lor(0<a<1 \wedge r<0), r<s\Leftrightarrow a^r<a^s$\\
Se $(0<a<1 \wedge r>0)\lor(a>1 \wedge r>0), r<s\Leftrightarrow a^r>a^s$\\
$\forall a\neq 1 a^r=a^s\Rightarrow r=s$
\section{Esponenziali $a^x$}
$a^x, a>1, x \in \mathbf{R}, x<0, x=p,\alpha_1\alpha_2\alpha_3\alpha_4, a^x\dot{=}sup\{a^{p,\alpha_1\alpha_2\alpha_3\alpha_4}\}$\\
Se $0<a<1, x\in \mathbf{R^+}, a^x\dot{=}\frac{1}{(\frac{1}{a})^x}$\\
Se $a>0, a\neq 1, x \in \mathbf{R^-}, a^x\dot{=}\frac{1}{a^{-x}}$\\
Rimangono soddisfatte tutte le propriet\`a elencate per $a^x$ con $x\in\mathbf{Q}$
\section{Logaritmi $\log_a x$}
Si intende per logaritmo la soluzione dell'equazione: $a^x=y, y\in \mathbf{R^+}$. La soluzione x si chiama algoritmo in base a di y ($x=\log_a y$). 
\subsection{Esistenza del logaritmo}
$a\in \mathbf{R}, a>0, a\neq 1, y\in \mathbf{R^+}\Rightarrow\exists!x\in\mathbf{R}:a^x=y$.\\
La dimostrazione si basa sull'esistenza dell'estremo superiore e procede in maniera analoga all'esistenza della radice n-essima,
\subsection{Propriet\`a}
\begin{itemize}
\item $a^{\log_a y}=y,\forall y>0$
\item $\log_a a =1$
\item $\log_a(x_1\cdot x_2)=\log_a x_1 + \log_a x_2 \forall x_1, x_2 \in \mathbf{R}$
\item $\log_a(\frac{x_1}{x_2})=\log_a x_1 - \log_a x_2 \forall x_1, x_2 \in \mathbf{R}$
\item $\log_a a^{\alpha} =\alpha\log_a a, \forall x>0, \alpha \in \mathbf{R}$
\item $\log_a a^x =x \forall x\in \mathbf{R}$
\item $\log_a 1 =0$
\item $\log_a x =-\log_{\frac{1}{a}} x$
\item Cambio di base: $\log_b x=\frac{\log_a x}{\log_a b}$
\item $0<x_1<x_2$
\begin{itemize}
\item $\Leftrightarrow \log_a x_1<\log_a x_2, a>1$
\item $\Leftrightarrow \log_a x_1>\log_a x_2, 0<a<1$
\end{itemize}
\end{itemize}
\chapter{I numeri complessi}
I numeri complessi vengono introdotti come soluzioni di equazioni algebriche del tipo $x^2+1=0$. Si introduce perci\`o l'unita immaginaria \emph{i}, 
in modo che $i^2+-$. Pertanto l'insieme dei numeri complessi $\mathbf{C}$ in forma algebrica (o cartesiana), viene definito come 
$\mathbf{C}\dot{=}\{z=x+iy, x, y\in\mathbf{R}$, con x parte reale, o Rez e i parte immaginaria o Imz, $\mathbf{R}\subset\mathbf{C}$. I numeri reali sono 
i numeri complessi con parte immaginaria uguale a 0. 
\section{Operazioni e propriet\`a in $\mathbf{C}$}
$z_1=x_1+iy_1$, $z_2=x_2+iy_2$
\begin{itemize}
\item In $\mathbf{C}$ non esiste ordinamento
\item Somma: $z_1+z_2=x_1+x_2+i(y_1+y_2)$
\begin{itemize}
\item Elemento neutro della somma $z_0=0+i0$
\item Elemento opposto della somma $z_o=-x-iy$
\end{itemize}
\item Prodotto: $z_1\cdot z_2=x_1x_2-y_1y_2+i(y_1x_2+y_2x_1)$
\begin{itemize}
\item Elemento neutro del prodotto $z_0=1+0i$
\item Elemento reciproco del prodotto $\frac{1}{z}+\frac{x}{x^2+y^2}-i\frac{y}{x^2+y^2}$
\end{itemize}
\item Coniugato di $=x+iy$ \`e $\bar{z}=x-iy$
\begin{itemize}
\item $\bar{z_1+z_2}=\bar{z_1}+\bar{z_2}$
\item $\bar{z_1z_2}=\bar{z_1}\bar{z_2}$
\item $\bar{\bar{z}}=z$
\item $z\in\mathbf{R}\Leftrightarrow z=\bar{z}$
\end{itemize}
\item Modulo di z: $|z|\in\mathbf{R}, |z|=	\sqrt{x^2+y^2}$
\begin{itemize}
\item $z\in\mathbf{R}\Rightarrow|z|+=\sqrt{x^2}=|x|$
\item $|z|=|\bar{z}|$
\item $z\cdot\bar{z}=|z|^2$
\item $\forall z\in\mathbf{R}, z\neq 0, \frac{1}{z}=\frac{\bar{z}}{|z|^2}$
\end{itemize}
\end{itemize}
\section{Forma trigonometrica di un numero complesso}
Un numero complesso si esprime in forma trigonometrica attraverso il suo modulo e l'angolo che l'asse reale forma con la retta passante per l'origine e il numero nel piano di 
Gauss, detto argomento del numero (argz). 
L'angolo non \`e univoco, ma viene espresso attraverso il numero periodico $\theta+2k\pi$.\\
La parte reale $x=Rez=|z|\cos(\theta+2k\pi)$; la parte immaginaria $y=Imz=|z|\sin(\theta+2k\pi)$\\
Il numero si dice pertanto in forma trigonometrica quando \`e nella forma $z=\rho(\cos(\theta+2k\pi)+\sin(\theta+2k\pi)i$
\section{Potenza n-esima di $z\in\mathbf{C}$}
Posto $z=|z|(\cos(argz +2k\pi)+\sin(argz +2k\pi)i)$, secondo la legge di de Moivre($(\cos\theta+\sin\theta)^n=(\cos n\theta+\sin n\theta)$), $z^n=|z|^n(\cos(n\cdot argz +2k\pi)+\sin(n\cdot argz +2k\pi)i), k\in\mathbf{Z}$.
\section{Radici n-esime di $w\in\mathbf{C}$}
Lo scopo \`e trovare le soluzioni z dell'equazione $z^n=w$. Dato un numero reale w, se $w=0$, la sola radice n-esima di w\`e 0. Se $w\neq 0$ ha n radici n-esime distinte date
da: $z_k=\sqrt[n]{|w|}(\cos(\frac{argw}{n}+\frac{2k\pi}{n})+\sin(\frac{argw}{n}+\frac{2k\pi}{n})i), k\in\mathbf{N}\wedge k=0,1,2,\cdots,n-1$. Tutte le radici ennesime si 
trovano lungo una circonferenza e formano il poligono regolare con n lati inscritto a tale circonferenza di raggio $\sqrt[n]{|w|}$.
\section{Forma esponenziale di un numero complesso}
$e^{i\theta}\dot{=}\cos\theta+\sin\theta i$, perci\`o $\forall z\in\mathbf{C} z=|z|e^{i(argz+2k\pi})$
\begin{itemize}
\item $z_1\cdot z_2=|z_1||z_2|e^{i(argz_1+argz_2)}$
\item $z^n=|z|^ne^{inargz}$
\item $e^{i\pi}=-1$
\end{itemize}