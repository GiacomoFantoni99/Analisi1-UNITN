\chapter{Numeri Complessi}
\section{Definizione}
Viene definito unità immaginaria $i$ quel numero per cui $i^2 = -1$.
L'insieme dei numeri complessi è quindi:\\
\begin{Large}
$\mathbb{C} = \{z = x + iy;\ x,y \in \mathbb{R}\}$
\end{Large}\\
ove $x + iy$ è la scrittura algebrica di un numero complesso.\\
In particolare $x$ è chiamata parte reale del numero: $x = Rez$; mentre $y$ parte immaginaria $y = Imz$.\\
\subsubsection{Osservazione}
$z \in \mathbb{C}$ è reale ($z \in \mathbb{R}$) $\iff Imz = 0$

\section{Piano di Gauss}
Il piano di Gauss è un piano $\mathbb{R}^2$ ove l'asse delle ascisse è $Rez$ e l'asse delle ordinate $Imz$\\
Ogni numero complesso $z$ è quindi rappresentabile come un punto sul piano di Gauss\\
---------------TBD: INSERIRE IMMAGINE PIANO DI GAUSS--------------------

\section{Alcune operazioni}
$z_1,z_2 \in \mathbb{C}$
\subsection{Somma}
\begin{Large}
$z_1+z_2 = x_1+x_2+i(y_1+y_2)$ 
\end{Large}\\
$Re(z_1+z_2) = x_1+x_2$;   $Im(z_1+z_2) = y_1+y_2$
\subsection{Prodotto}
\begin{Large}
$z_1z_2 = x_1x_2-y_1y_2+i(x_1y_2+ x_2y_1)$ 
\end{Large}\\
$Re(z_1z_2) = x_1x_2-y_1y_2$;   $Im(z_1z_2) = x_1y_2 + x_2y_1$
\subsection{Modulo}
$z = x+iy \in \mathbb{C}$\\
Modulo di $z$:\\
\begin{Large}
$|z| = \sqrt{x^2+y^2} \in \mathbb{R}$
\end{Large}\\
Nel piano di Gauss rappresenta la distanza dall'origine

\section{Numeri complessi notevoli}
$z \in \mathbb{C}$
\subsubsection{Elemento neutro somma}
$0\ +\ i0$
\subsubsection{Elemento opposto somma}
$-x\ -\ iy$
\subsubsection{Elemento neutro prodotto}
$1\ +\ i0$
\subsubsection{Elemento reciproco prodotto}
\begin{Large}
$\frac{1}{z}=\frac{1}{x+iy}=\frac{x-iy}{x^2+y^2}$
\end{Large}

\subsection{Coniugato di z}
$\bar{z} = x-iy$
\subsubsection{Osservazioni}
$z_1,z_2 \in \mathbb{C}$\\
$\overline{1+z_2} = \overline{z_1}+\overline{z_2}\\
\overline{z_1z_2} = \overline{z_1}\overline{z_2}$\\
\\
\begin{Large}
$|\bar{z}| = |z|\\
z\bar{z}=|z|^2\\
\frac{1}{z} = \frac{\bar{z}}{|z|^2}$
\end{Large}

\section{Rappresentazione trigonometrica}
$z = x+iy$\\
Un numero complesso è univoco rispetto la sua distanza dall'origine del piano di Gauss e l'angolo con l'asse delle ascisse, è quindi rappresentabile in funzione di questi due fattori:\\
\begin{Large}
$z=r(cos(\theta)+isen(\theta)) =\\
r(cos(\theta+2k\pi)+isen(\theta+2k\pi)) \forall k \in \mathbb{Z}$
\end{Large}\\
---------------TBD: INSERIRE GRAFICO--------------------\\
In particolare:\\
\begin{Large}
$r = |z| \geq 0\\
\theta$ è detto \textbf{argomento} di $z$
\end{Large}

\section{Operazioni con rappresentazione trigonometrica}
\subsection{Prodotto}
$z_1,z_2 \in \mathbb{C}$;  $\theta_1 = argz_1,\ \theta_2=argz_2$\\
\begin{Large}
$z_1z_2=|z_1||z_2|(cos(\theta_1+\theta_2)+isen(\theta_1+\theta_2))$
\end{Large}
\subsubsection{Dimostrazione}
$z_1z_2=|z_1|(cos(\theta_1)+isen(\theta_1)) * |z_2|(cos(\theta_2)+isen(\theta_2))=\\
|z_1||z_2|(cos(\theta_1)cos(\theta_2) + i^2sen(\theta_1)sen(\theta_2) + icos(\theta_1)sen(\theta_2) + isen(\theta_1)cos(\theta_2)) =\\
|z_1||z_2|((cos(\theta_1)cos(\theta_2)-sen(\theta_1)sen(\theta_2)) + i(cos(\theta_1)sen(\theta_2) + sen(\theta_1)cos(\theta_2))) =
z_1z_2=|z_1||z_2|(cos(\theta_1+\theta_2)+isen(\theta_1+\theta_2))$
\subsection{Potenza - formula di De Moivre}
$z \in \mathbb{C}$\\
\begin{Large}
$z^n = |z|^n(cos(n\theta)+isen(n\theta))$
\end{Large}
\subsection{Radici n-esime}
$w,z \in \mathbb{C}$\; $n \in \mathbb{N}$\\
Se $n>1$, un numero $z$ è radice n-esima di $w \iff z^n=w$\\
Sia $w \in \mathbb{C}, n \in \mathbb{N}, n>1, w=|w|(cos(\theta)+isen(\theta)$:\\
\begin{enumerate}
\item[i.] Se $w=0$ la sola radice di $w$ è $z=0$
\item[ii.] Se $w\neq0$ allora $w$ ha \textbf{n} radici \textbf{distinte} che sono:\\
\begin{Large}
$z_k=\sqrt[n]{|w|}(cos(\frac{\theta+2k\pi}{n})+isen(\frac{\theta+2k\pi}{n})),\ k \in \mathbb{N},\ 0\leq k<n$
\end{Large}
\end{enumerate}
\subsubsection{Dimostrazione}
\begin{enumerate}
\item[i.]$w=0 \iff |z^n|=0 \iff |z|=0 \iff z=0$
\item[ii.]$z=|z|(cos(t)+isen(t))$; $z^n=w \iff \\
|z|^n(cos(nt)+isen(nt)) = |w|(cos(\theta+2k\pi)+isen(\theta +2k\pi) \iff
\begin{cases}
|z|^n = |w|\\
nt = \theta +2k\pi
\end{cases}$
\end{enumerate}
\subsection{Teorema fondamentale dell'algebra}
L'equazione polinomiale della forma: $a_nz^n + a_{n-1}z^{n-1}+...+a_1z+a_0$, $a_n \neq 0$, $a_i \in \mathbb{C}$ ha esattamente \textbf{n} radici in $\mathbb{C}$

\subsection{Notazione Esponenziale}
Per la formula di eulero:\\
$e^{i\theta} = cos(\theta)+isen(\theta)$\\
Un numero complesso $z = |z|(cos(\theta)+isen(\theta))$ può quindi essere scritto come:\\
\begin{Large}
$z = |z|e^{i\theta}$
\end{Large}\\
denominata notazione esponenziale.